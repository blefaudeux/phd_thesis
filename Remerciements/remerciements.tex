Difficile position que celle des remerciements, premiers dans ce manuscrit tout en étant tardifs; car ils sont consécutifs à toutes les aides, encouragements, remarques et critiques qui lui ont donné naissance. On court ainsi le risque dans cet exercice d'arriver trop tard, alors que les débuts de la thèse sont déjà loin, et que les premiers encouragements et discussions, si importants dans le résultat final, sont oubliés. Qu'il me soit cependant permis d'espérer, comme tant d'autres, profiter de ce moment pour remercier toutes les personnes qui m'ont beaucoup aidé tout au long de cette période.\\
L'exercice est habituel, mais je voudrais tout d'abord remercier les membres du jury, et tout particulièrement mes rapporteurs, pour leur relecture attentive et leurs nombreuses remarques. J'y ai trouvé une source de réflexion, des corrections efficaces, des remarques encourageantes et qui pourront dans l'avenir découler sur un travail complémentaire sur les nombreux points à approfondir. Ces remerciements étant postérieurs à ma soutenance, je souhaiterais également remercier les membres du jury pour leurs questions et critiques, qui m'ont offert l'occasion de défendre mon travail et d'en comprendre certaines des limites. Ces questions étaient très intéressantes, les points de désaccord persistants étant à mon sens fructueux, j'y trouve des éléments de critique qui me semblent indispensables.\\

Je ne serais pas arrivé à cette étape sans l'aide précieuse de nombreux relecteurs, qui ont su faire d'un manuscrit très perfectible une version au moins acceptable telle qu'elle est ici introduite. Merci François, Anne-Sophie, Évangéline, Jorge, Raoul pour votre relecture approfondie et vos très nombreuses remarques, les meilleures parties de ce manuscrit vous sont fortement redevables. Je n'y serais pas non plus arrivé sans l'équipe d'IMARA et celle du CAOR, qui m'ont accueilli pendant ces trois ans. Bien que n'étant que trop peu présent, l'équipe d'IMARA m'a toujours offert un cadre chaleureux et les moyens de tester et de faire avancer les idées présentes dans ce manuscrit. Merci François et Armand pour le support technique ! L'équipe du CAOR, qui m'aura fourni un cadre de travail appréciable et apprécié, a aussi rendu possible d'innombrables discussions avec Anne-Sophie, Raoul et Amaury (pour n'en citer que quelques uns) sans qui je n'auras bien souvent pas su avancer. Merci Christine et Christophe pour votre aide lors des moments clefs, lors des petits retards et grands livrables, pour votre bonne humeur\\

Au delà d'un cadre de travail, une thèse consiste également en une interaction particulière avec une personne, référent et organisateur d'un travail au long cours. Merci Fawzi pour ton soutien, pour le temps que tu as su prendre, lors de ma rédaction pour organiser les idées et le manuscrit et tout au long de ma thèse ; pour tes suggestions, demandes et remarques pendant mes recherches.\\

Merci enfin à ma famille et à mes amis, pour m'avoir supporté pendant ces trois années, tout en vous demandant parfois ce que je pouvais bien chercher... Merci de ne pas me prendre au sérieux, de rire ou de vous moquer de mes bêtises et remarques, de m'apporter un point de vue extérieur qui m'est précieux. Votre soutien compte beaucoup pour moi. Merci enfin Marie pour ta présence, ta confiance et tes encouragements. Merci de supporter tout mes défauts, et de me pousser vers le meilleur.\\