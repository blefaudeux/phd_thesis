\section{Conclusion} 
% 1 - Rappel des enjeux
% 2 - On revient sur la solution proposée

% Remettre cette étape de perception en perspective par rapport à toutes les tâches gérées en robotique
La perception de l'environnement est l'une des multiples tâches qu'un véhicule autonome doit effectuer, parallèlement à d'autres actions souvent plus visibles du grand public. La communication, l'intelligence artificielle ou les capacités motrices sont en effet aisément mises en avant, mais la connaissance du milieu dans lequel un robot évolue ne doit pas être négligée. La prise de décision autonome, qui caractérise notamment un véritable robot indépendant par rapport à un automate simplifié, n'est en effet pas possible sans éléments de choix, sans connaissance préalable. La réaction autonome d'un porteur par rapport à son environnement est donc largement tributaire de sa perception de celui-ci.\\
% Plus d'une décennie de travaux sur les SLAM, mais il y a encore du boulot
La localisation et la cartographie simultanée ont fait l'objet de nombreuses recherches pendant la décennie passée, en exploitant des capteurs divers, tels que des télémètres lasers, des caméras classiques ou adaptées à une captation en trois dimensions. Connus sous le nom de SLAM, ces algorithmes rendent possible le déplacement d'un robot au sein d'un environnement initialement inconnu, mais aussi dans certains cas sa localisation par rapport à une carte déjà acquise. Il s'agit donc tout à la fois d'un moyen de localisation et de cartographie de son environnement, susceptible de prendre en compte les modifications récentes de celui-ci. Ce travail n'est pas terminé, la cartographie à grande échelle pouvant encore être pris en défaut de cohérence (le fameux problème de la fermeture de boucle), tandis que la cartographie reconstruite n'offre en général qu'une image partielle de l'environnement exploré, car peu dense ou selon un nombre restreint de dimensions. Un autre aspect reste cependant à développer, et il s'agit celui que nous avons tenté de mettre en avant dans ce manuscrit. \\
% .. particulièrement en ce qui concerne les objets mobiles
La perception des éléments mobiles est, en effet, elle aussi indispensable à tout véhicule autonome, particulièrement dans un environnement incontrôlé et dynamique comme peuvent l'être les lieux d'activité humaine. Un véhicule réellement \og intelligent\fg{} doit ainsi prendre en compte son environnement futur dans ses déplacements, seul moyen d'obtenir des trajectoires à la fois sécurisées et efficaces. Cet aspect de la perception est également très présent dans l'état de l'art, notamment suite à l'article fondateur de Wang \textit{et al.} (\cite{Wang2007}), mais de nombreux travaux restent en cours.\\

% Retour rapide sur la méthode
La méthode proposée exploite la vision pour estimer, en temps réel, la position d'un important échantillon de points caractéristiques de l'environnement. Le nombre de points positionnés, bien qu'inférieur à celui proposé par les méthodes considérant un environnement statique, est bien supérieur à celui de nombreuses techniques existantes dans l'état de l'art prenant en compte un environnement dynamique. Mettant l'accent sur la perception des obstacles potentiels, cette méthode autorise un compromis sur la précision de localisation des points pour s'assurer que leur nombre sera suffisant. Nous proposons ensuite une approche exploitant une technique de segmentation multi-dimensionnelle, exploitant en cela la nature temporelle et mobile du nuage de point positionné. Un algorithme de filtrage et de suivi de cibles multiples, présent dans l'état de l'art mais appliqué pour la première fois à la détection et au suivi de cibles mobiles depuis un dispositif visuel en mouvement, est finalement proposé pour maximiser les informations disponibles au sortir de cette étape de perception.\\

% Ouverture ?
La problématique initiale d'une proposition compatible avec une exécution en temps réel est respectée par cet algorithme, tandis que son domaine d'application est ouvert aux déplacements rapides, comme le montrent des tests à partir d'une plate-forme automobile en environnement urbain. Ce système impose cependant des limites en termes de domaine de perception, étant bien sûr sensible à la qualité des acquisitions visuelles et à la présence de points caractéristiques dans l'environnement. La fusion d'informations avec d'autres capteurs aux caractéristiques complémentaires, son intégration dans un système plus global à même de répondre aux problématiques de cartographie et de localisation, restent ainsi des pistes que nous souhaiterions développer.

\section{Perspectives} 
% Coupler avec de la cartographie
Il serait ainsi possible d'associer cette méthode, centrée sur la détection d'obstacles et le suivi d'objets mobiles, à un algorithme de cartographie et d'optimisation globale de l'environnement reconstitué. De telles méthodes font l'objet de nombreux travaux actuellement, et étendraient les informations extraites visuellement à d'autres applications.\\
% Fusion avec plusieurs capteurs
Les limitations présentées par notre méthode, en termes de mouvements de plate-forme acceptés, de portée, ou de problème de perception dans un environnement inadapté (sombre ou soumis à des aléas climatiques), accentuent par ailleurs l'intérêt des dispositifs algorithmiques de fusion multi-capteurs. Il s'agit là encore d'une problématique activement explorée de nos jours, et qui constituerait un développement à notre sens très intéressant de la méthode proposée. \\
% Collaboration entre plusieurs véhicules
On pourra enfin revenir sur les applications rendues possibles par une perception distribuée sur plusieurs porteurs, dans le sens où celle-ci serait partagée par chacun d'entre eux. La popularisation d'engins aériens autonomes et de petite taille en fournit une application immédiate, mais les bénéfices d'une perception collaborative seraient tout aussi patents pour des véhicules terrestres, en termes de portée ou d'observabilité par exemple. Les développements récents des communications à haut débit entre plusieurs nœuds de communication mobiles rendent ce développement réaliste, mais il s'agit d'une problématique complexe, et son cadre algorithmique reste majoritairement en devenir.